\documentclass[12pt]{gWmaths}
%!TEX program = pdflatexmk
%!TEX encoding = UTF-8
%!TEX spellcheck = en-US
%%%%%%%%%%%%%%%%%%%%%%%%%%%%%%%%%%%%%%%%%%%%%%



%%
\setlength{\oddsidemargin}{0in}
\setlength{\evensidemargin}{0in}
\setlength{\textwidth}{6.5in}
\setlength{\parindent}{0in}
%\setlength{\parskip}{\baselineskip}
%\setlength{\parskip}{0em}
%\setlength{\unitlength}{1in}
%%
\addtolength{\textwidth}{100pt}
\addtolength{\evensidemargin}{-60pt}
\addtolength{\oddsidemargin}{-60pt}
\addtolength{\topmargin}{-70pt}
\addtolength{\textheight}{2in}
%\setlength{\parindent}{0in}
%\setlength{\parskip}{10pt}
%%
\linespread{1}
%%

%%%%%%%%%%%%%%%%%%%%%%%%%%%%%%%%%%%%%%%%%%%%%%
% Now we're ready to start
%%%%%%%%%%%%%%%%%%%%%%%%%%%%%%%%%%%%%%%%%%%%%%
\author{John Pell}
\title{Accessible Tagged PDFs for Mathematics}
\date{\today}



\begin{document}
\maketitle

\section{Introduction}
\paragraph{Show all your work!} Little or no credit will be given for correct answers that lack sufficient justification. You may use a calculator but no other electronic devices are allowed. Simplify whenever possible.

\section{testing...}

\begin{theorem}[10 pts]
Let $U = \{1,2,3,4,5,6,7,8,9,10 \}, X = \{2,3,7 \},Y = \{1,4,5,6,8 \}, \\
Z = \{2,3,7,9 \}$. List the elements of the following sets. 
\end{theorem}

\Large


\begin{Exercise}
Let $$f(x) = \left\{
\begin{array}{ll} 
 x^2 + 2 & \text{if } x < 0 \\
\sin x & \text{if } x \geq 0
\end{array} \right. $$

\begin{enumerate}[a)]
\item Graph $f(x)$ below. \\

%\resizebox{4.5in}{4.5in}{\includegraphics{p3.pdf}}

\centerline{\bf TURN OVER}
\vfill\eject

\noindent
\textbf{In parts (b) - (d), find the limit. If the limit does not exist, write DNE and explain why.} \\

\item  $\lim_{x \to 0^-} f(x)$
\vspace{0.3 in}
\item  $\lim_{x \to 0^+} f(x)$
\vspace{0.3 in}
\item  $\lim_{x \to 0} f(x)$
\vspace{0.5 in}
\item Is $f(x)$ continuous at $x = 0$? If not, is it left-continuous or right-continuous? Justify your answer using definition of continuity. 
\vspace{2 in}
\end{enumerate}
\end{Exercise}

\begin{Exercise}
Find the derivative of the function using the limit definition of derivative. $$f(x) = \frac{1}{\sqrt{x}}$$
\vspace{8 in}
\end{Exercise}

\centerline{\bf TURN OVER}
\vfill\eject


\begin{Exercise}
Find $\frac{dy}{dx}$ by implicit differentiation and evaluate the derivative at point $(-1, 1)$.
$$(2x + 2y)^3 = 8x^3 + 8y^3$$
\vspace{0.3 in}

$\frac{dy}{dx} = -\frac{y(y + 2x)}{x (x + 2y)}$ \\

At $(-1, 1)$, $y' = -1$
\end{Exercise}

\vspace{0.3 in}


\begin{Exercise}
Find the limit. For each part, thoroughly explain how you arrived at the answer!!! 
\begin{enumerate}[a)]
\item  $\lim_{x \to \infty} e^{x - x^2}$
\vspace{4 in}
\item  $\lim_{x \to -\infty} \frac{\sqrt{1 + 4x^6}}{2 - x^3}$
\vspace{4 in}
\end{enumerate}
\end{Exercise}

\centerline{\bf DONE!}
\vfill\eject




\noindent
\textbf{Section 12.6: The Normal Distribution}\\


\begin{Def}
A \textbf{normal distribution} is a continuous, symmetric, bell-shaped distribution. \\
\end{Def}

%\includegraphics[width=10cm]{Bell Shaped.jpg} \\

\noindent
\textbf{}

\begin{enumerate}[1.]
\item It is bell-shaped.
\item The mean, median, and mode are all exactly the same, and are located at the center of the distribution.
\item It's symmetric about its mean. In other words, if you draw a vertical line through the center, the graph is divided into two identical halves.
\item The curve is continuous; it has no gaps or holes, and extends from $-\infty$ to $+\infty$ along the horizontal axis.
\item The area under any portion of the curve is the percentage (in decimal form) of data values that fall between the values that begin and end the region. (We'll use this property a LOT in Section 12 -7.)
\item The total area under the entire curve is 1. This makes sense based on properties 4 and 5: $100\%$ of the data values are somewhere on the real number line.
\end{enumerate}

\vspace{1 in}

\noindent
\textbf{The Empirical Rule}\\

\begin{Def}
When data are normally distributed, approximately $68\%$ of the values are within 1 standard deviation of the mean, approximately $95\%$ are within 2 standard deviations of the mean, and approximately $99.7\%$ are within 3 standard deviations of the mean (see Figure below).\\
\end{Def}


%\includegraphics[width=15cm]{Empirical Rule.jpg} \\


\noindent
\textbf{The Standard Normal Distribution}\\

\begin{Def}
The \textbf{standard normal distribution} is a normal distribution with mean 0 and standard deviation 1. \\
\end{Def}

%\includegraphics[width=10cm]{SND.jpg} \\

\noindent
For a given data value from a data set that is normally distributed, we define that value's \textbf{$z$ score} to be\\ $$z = \frac{\text{ Data value - mean } }{\text{ Standard deviation }} = \frac{x - \mu}{\sigma}$$ \\


\begin{Exercise}
A standard test of intelligence is scaled so that the mean IQ is 100, and the standard deviation is 15. Find the z score for a person with an IQ of 91.
\vspace{10 in}
\end{Exercise}


\noindent
\textbf{Finding Areas under the Standard Normal Distribution}\\

\noindent
Two Important Facts about the Standard Normal Curve

\begin{enumerate}[1.]
\item The area under any normal curve is divided into two equal halves at the mean. Each of the halves has area 0.500.
\item The area between $z = 0$ and a positive $z$ score is the same as the area between $z = 0$ and the negative of that $z$ score.
\end{enumerate}


\begin{Exercise}
Find the area under the standard normal distribution 
\begin{enumerate}[a.]
\item Between z = 1.55 and z = 2.25.
\vspace{10 in}

%\includegraphics[width=18cm]{Z.jpg} \\

\item Between z = - 0.60 and z = -1.35.
\vspace{10 in}


\item Between z = 1.50 and z = -1.75.
\vspace{10 in}

%\includegraphics[width=18cm]{Z.jpg}

\end{enumerate}
\end{Exercise}



\begin{Exercise}
Find the area under the standard normal distribution 
\begin{enumerate}[a.]
\item To the right of z = 1.70.
\vspace{10 in}

\item To the right of z = -0.95.
\vspace{3 in}
\end{enumerate}
\end{Exercise}

%\includegraphics[width=18cm]{Z.jpg} 

\begin{Exercise}
Find the area under the standard normal distribution 
\begin{enumerate}[a.]
\item To the right of z = - 2.40.
\vspace{10 in}
\item To the right of z = 0.25.
\vspace{10 in}

%\includegraphics[width=18cm]{Z.jpg} 

\item To the left of z = -2.20.
\vspace{10 in}
\end{enumerate}
\end{Exercise}



\end{document}
