\documentclass[english,lecture]{gWmaths}
%%%%%%%%%%%%%%%%%%%%%%%%%%%%%%%%%%%%%%%%%%%%%%
%% Now we're ready to start
%%%%%%%%%%%%%%%%%%%%%%%%%%%%%%%%%%%%%%%%%%%%%%
%%
\author{Diana Pell}
\title{TITLE OF DOCUMENT}
\date{Fall 2020}  % \LaTeXe* uses \today if you don't specify any \date{}
%%
\addto\captionsenglish{%
	\renewcommand{\abstractname}{What To Expect}% Heading of Intro section
}
%%


\tagpdfsetup{}
\hypersetup{
	colorlinks=true,
	linkcolor=blue,
	filecolor=magenta,
	urlcolor=cyan,
		}%\hypersetup
\urlstyle{
	%same, % Use the 'same' style as surrounding text, default is monospace font
		}%\urlstyle
\bookmarksetup{}
\geometry{}
\graphicspath{}

\GetFileInfo{\CurrentFile}
\title{\fileinfo}
\date{\filedate}



\begin{document}


\end{document}




%<*xmplmain|xmpllecture|xmplexam|xmplquiz|xmplgroup>%%%%%%%%%%%%%%%%%%%%%%%%%%%%
%    \begin{macrocode}
\documentclass[%
%%!TEX encoding = UTF-8
%%!TEX parameter = -8bit
	main=english,% The language the course is taught in, used by \pkg{babel}
%%!TEX spellcheck = en-US
	course=MATH-101,% Put the official course designation here, and the next line:
%%!TEX root = MATH-101.TEX
	catalognumber=15243,% Put the catalog/section number for this iteration of the class.
%<*xmpllecture|xmpl_stuff_>
	chapters=1,% This lecture only covers content from Chapter 1 of the text book.
	firstsection=2,% This lecture begins with section 2 of the chapter.
	% See below for how to skip sections
%</xmpllecture|xmpl_stuff_>
]{gWmaths}
%%!TeX program = xelatexmk
%    \end{macrocode}
\author{Professor Sarah Connor}
%<xmplmain>\title{Mathematics for Liberal Artists}% Title of the course
%<xmpllecture>\title{Sets}% Title of the chapter
%<xmpl_stuff_>% title of the quiz/exam/coverage/whatshit
%%
%<xmplmain>% First Day of Class, used to generate "Spring 2202" and weekly schedule
%<xmpllecture>% Date this lecture is to be delivered, parsed and formatted as "Week 1"
%<xmpl_stuff_>%
\date{2022-02-02}
%%
\begin{document}% Actual content begins here:
\maketitle% Will set \thechapter, set \thechaptermark, and print a title

% some other stuff
%%

%<*xmpllecture>
\section{Using Venn Diagrams to Study Set Operations}% Increments section counter and prints a heading

\subsection{Finding the Difference of Two Sets}

\begin{Exercise}
Let $A = \{4, 6, 8, 10 \}, B = \{2, 6, 12\}$, and $C = \{8, 10 \}$. Find each set.
\begin{enumerate}[a)]
\item $A - B$
\vspace{0.5 in}
\item $B - C$
\vspace{0.3 in}
\item $(A - B) - C$
\vspace{0.5 in}
\end{enumerate}
\end{Exercise}
%</xmpllecture>



% some other stuff
\end{document}
%</xmplmain|xmpllecture|xmplexam|xmplquiz|xmplgroup>
