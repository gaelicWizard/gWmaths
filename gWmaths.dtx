\NeedsTeXFormat{LaTeX2e}[2020/10/01]%
%\begin{abstract}
%<*readme>
%% During the great pandemic of 2020, \cls{gWmaths} class was created to shorten the common preamble from a set of my wife's \LaTeXe\ lectures, transparencies, slides, and handouts. It was initially \cls{pellmaths} and was just a shortcut for 80 lines of boilerplate at the top of every darn file.
%% With the first week of 2021 declaring "hold my beer", some effort was made to expand and improve. The class was renamed to \cls{gWmaths.cls}, partially split out to a \pkg{gWmaths.sty} package, and updated to \LaTeX3. Meanwhile, the source transformed to require build via \pkg{docstrip} and was brought under version control. \cs{tehMacro} %\marg{m} \oarg{p} \parg{picture mode argument}. \file{\jobname} / \env{\jobname} / \pkg{\jobname} / \cls{\jobname}. \NB{tehNote}{some notes}
%%
%</readme>
%\end{abstract}
\RequirePackage{ iftex, etoolbox,	}%
%%%%%%%%%%%%%%%%%%%%%%%%%%%%%%%%%%%%%%%%%%%%%%%%%%%%%%%%%%%%%%%%%%%%%%%%%%%%%%%%
%    \begin{macrocode}
%<*ins>
\RequirePDFTeX % ASCII 9 (tab) becomes \TeX ^^I in \XeTeX
\def\@docextension{ltx}
\begingroup	\input l3docstrip	\ifToplevel{%
	\askforoverwritefalse	%\keepsilent	%
	\usedir{tex/latex/\jobname}	\UseTDS	%
	\nopreamble	\nopostamble	}%
\generate{	\catcode9=12 % tabs %https://tex.stackexchange.com/a/453323
  \file{README.md}{\from{\jobname.dtx}{readme}}
  \filename@parse{\jobname.ins} % https://tex.stackexchange.com/a/39636 https://tex.stackexchange.com/a/39647
  \file{\filename@base.\filename@ext}{\from{\filename@base.dtx}{\filename@ext}}
  \file{\jobname.\@pkgextension}{\from{\jobname.dtx}{\@pkgextension}}
  \file{\jobname.\@clsextension}{\from{\jobname.dtx}{\@clsextension}}
  \file{\jobname.\@docextension}{\from{\jobname.dtx}{\@docextension}}
}	\ifToplevel{	\ReportTotals
 % \@currext \@currname \@currenvir \@currenvline \@currentlabel %\@onlypreamble{\tehMacro} % \g@addto@macro\tehMacro{\tehStuff{}}
 % TODO: figure out how to make <*ins> open an environment and </ins> close it
%<ins>\endbatchfile
}	\endgroup	% \end{docstrip}
%%%%%%%%%%%%%%%%%%%%%%%%%%%%%%%%%%%%%%%%%%%%%%%%%%%%%%%%%%%%%%%%%%%%%%%%%%%%%%%%
%</ins>
%<cls>\ProvidesExplClass%
%<sty>\ProvidesExplPackage%
%<*!sty&!cls>
\ProvidesExplFile%
%</!sty&!cls>
 	{gWmaths}	{2021/01/14}	{0.3.0}%
 		{COVID-19 Quarantine Accessible TeX Project}
%<*tex>
\documentclass{l3doc}
\usepackage{gWmaths}
\AtBeginDocument{\OnlyDescription}
%%\AtBeginDocument{\CodelineIndex\EnableCrossrefs\RecordChanges}
\GetFileInfo{\jobname}
	\title{\fileinfo}
	\date{\filedate}
	\author{\pkg{\filename}~\thanks{\url{https://github.com/gaelicWizard/gWmaths}}~\fileversion}
\AtEndPreamble{\ExplSyntaxOff}
\begin{document}
	\maketitle
\begingroup	\catcode`\<=\catcode`\%
	\DocInput{\jobname.dtx}
	\ExplSyntaxOn
	%\clist_map_inline:cn {g_gWpkgs_clist} {\file{\jobname}~requires~\pkg{#1}. }
	\file{\jobname}~requires~the~following~packages:~\clist_use:cn {g_gWpkgs_clist}{,~}.
	%\clist_show:c {g_gWpkgs_clist}
	\ExplSyntaxOff
	%\IndexInput{}
	%\MakeShortVerb{\<}
\endgroup
\end{document}
%    \end{macrocode}
%%%%%%%%%%%%%%%%%%%%%%%%%%%%%%%%%%%%%%%%%%%%%%%%%%%%%%%%%%%%%%%%%%%%%%%%%%%%%%%%
%</tex>
% \StopEventually{END OF LINE} % must put \Finale at the bottom
%<*dtx>%%% If we're in the original DTX,
%% \endinput% then stop all further processing.
%</dtx>
%<*sty>
%%%%%%%%%%%%%%%%%%%%%%%%%%%%%%%%%%%%%%%%%%%%%%%%%%%%%%%%%%%%%%%%%%%%%%%%%%%%%%%%
%% Define logos for the \TeX family, w/ hyp
\ProvideDocumentCommand\TeX{}	{\hologo{TeX}\xspace}
\ProvideDocumentCommand\eTeX{}	{\hologo{eTeX}\xspace}
\ProvideDocumentCommand\XeTeX{}	{\hologo{XeTeX}\xspace}
\ProvideDocumentCommand\pdfTeX{}	{\hologo{pdfTeX}\xspace}
\ProvideDocumentCommand\LuaTeX{}	{\hologo{LuaTeX}\xspace}
\ProvideDocumentCommand\LaTeX{}	{\hologo{LaTeX}\xspace}
\ProvideDocumentCommand\LaTeXe{}	{\hologo{LaTeX2e}\xspace}
\ProvideDocumentCommand\LaTeXx{}	{\hologo{LaTeX3}\xspace}
 %% \Web \AmSTeX \BibTeX \SliTeX \PlainTeX
\AtEndPreamble{%
	\RequirePackage{hologo}%
		}%
%%%%%%%%%%%%%%%%%%%%%%%%%%%%%%%%%%%%%%%%%%%%%%%%%%%%%%%%%%%%%%%%%%%%%%%%%%%%%%%%
%% Fix up \NeedsTeXFormat to work from within {document}
\clist_gclear_new:c {g_gWpkgs_clist}
\AfterEndPreamble{%
\ifClassLoadedTF{l3doc}
{
	\RenewDocumentCommand\NeedsTeXFormat{ m O{\filedate} }
	{
		\file{\jobname}~requires~\hologo{#1},~dated~#2~or~newer.
	}
	\RenewDocumentCommand\RequirePackage{ o m O{\filedate} }
	{ % TODO: make \RequirePackage add packages to a list, rather than print
	%% \clist_new:N \g_docinput_clist
		\file{\jobname}~requires~~\pkg{#2},~dated~#3~or~newer.
	}
\RenewDocumentCommand \RequirePackage { o m o }
  {
    \clist_map_inline:nn {#2}
      {
        \clist_put_right:Nn \g_gWpkgs_clist {##1}
      }
  }
}
{}
}
%%%%%%%%%%%%%%%%%%%%%%%%%%%%%%%%%%%%%%%%%%%%%%%%%%%%%%%%%%%%%%%%%%%%%%%%%%%%%%%%
%% Hack up global package options list: 
\def\gW@classoptionslist{\@classoptionslist}
\providecommand\gW@addGlobalOption[1]{%
   \xdef\gW@classoptionslist{\gW@classoptionslist,#1}%
}%
%\AtEndOfClass{
%	%\let\gWoptions\gW@classoptionslist
%	\let\@classoptionslist\gW@classoptionslist
%		}
\AtEndPreamble{
%	\makeatletter
	\let\globalClassOptions\@classoptionslist
%	\makeatother
		}
%%%%%%%%%%%%%%%%%%%%%%%%%%%%%%%%%%%%%%%%%%%%%%%%%%%%%%%%%%%%%%%%%%%%%%%%%%%%%%%%
%% Define \GetFileInfo %%% Copy/pasta directly from `doc` package
\ProvideDocumentCommand\GetFileInfo{m}{%
  \edef\filename{#1}%
  \def\@tempb##1 ##2 ##3\relax##4\relax{%
    \def\filedate{##1}%
    \def\fileversion{##2}%
    \def\fileinfo{##3}}%
  \edef\@tempa{\csname ver@#1\endcsname}%
  \expandafter\@tempb\@tempa\relax? ? \relax\relax}
%%%%%%%%%%%%%%%%%%%%%%%%%%%%%%%%%%%%%%%%%%%%%%%%%%%%%%%%%%%%%%%%%%%%%%%%%%%%%%%%
%http://www.texfaq.org/FAQ-compjobnam
\def\jobis#1{FF\fi
  \edef\predicate{#1}%
  \edef\predicate{\expandafter\strip@prefix\meaning\predicate}%
  \edef\job{\jobname}%
  \ifx\job\predicate
}
%%%%%%%%%%%%%%%%%%%%%%%%%%%%%%%%%%%%%%%%%%%%%%%%%%%%%%%%%%%%%%%%%%%%%%%%%%%%%%%%
%% Define functions to get lists of classes, packages, files loaded % https://tex.stackexchange.com/a/43568
\seq_new:N \l_pclist_classes_seq
\seq_new:N \l_pclist_packages_seq
\seq_new:N \l_pclist_other_seq
\clist_map_inline:cn { @filelist }
  {
   \tl_if_in:nnTF { #1 } { .cls }
     {
      \tl_set:Nn \l_tmpa_tl { #1 }
      \tl_remove_once:Nn \l_tmpa_tl { .cls }
      \seq_put_right:NV \l_pclist_classes_seq \l_tmpa_tl
     }
     {
      \tl_if_in:nnTF { #1 } { .sty }
       {
        \tl_set:Nn \l_tmpa_tl { #1 }
        \tl_remove_once:Nn \l_tmpa_tl { .sty }
        \seq_put_right:NV \l_pclist_packages_seq \l_tmpa_tl
       }
       {
        \seq_put_right:Nn \l_pclist_other_seq { #1 }
       }
     }
  }
%%
\ProvideDocumentCommand{\ifClassLoadedTF}{ m m m }
{
	\seq_if_in:NnTF \l_pclist_classes_seq { #1 }
		{ %\msg_term:x { The~class~`#1'~is~loaded } 
		#2 }
		{ %\msg_term:x { The~class~`#1'~is~NOT~loaded } 
		#3 }
}
%%
\ProvideDocumentCommand{\showclassesloaded}{}{\seq_show:N \l_pclist_classes_seq}
\ProvideDocumentCommand{\showpackagesloaded}{}{\seq_show:N \l_pclist_packages_seq}
\ProvideDocumentCommand{\showfilesloaded}{}{\seq_show:N \l_pclist_other_seq}
%%%%%%%%%%%%%%%%%%%%%%%%%%%%%%%%%%%%%%%%%%%%%%%%%%%%%%%%%%%%%%%%%%%%%%%%%%%%%%%%
%% Define \ifLaTeX3
% \begin{macro}{\ifLaTeXxTF}
% \begin{arguments}
%   \item Code to run if the format is \LaTeXx.
%   \item Code to run if the format is not \LaTeXx.
% \end{arguments}
% Note that this macro is defined in \LaTeXx{} syntax, so failure is impossible.
%   \begin{macrocode}
\ProvideDocumentCommand \ifLaTeXxTF { O{} O{} }
{
	\ifcsdef {ExplSyntaxOn} {#1} {#2}
}
%   \end{macrocode}
% \end{macro}
%%%%%%%%%%%%%%%%%%%%%%%%%%%%%%%%%%%%%%%%%%%%%%%%%%%%%%%%%%%%%%%%%%%%%%%%%%%%%%%%
%% 
\ProvideDocumentCommand\mail{m}
{
	\href{mailto:#1}{\texttt{#1}}
}
%%%%%%%%%%%%%%%%%%%%%%%%%%%%%%%%%%%%%%%%%%%%%%%%%%%%%%%%%%%%%%%%%%%%%%%%%%%%%%%%
%% Hack up loaded package list 
% https://tex.stackexchange.com/a/491456
\expandafter\def\csname ver@l3regex.sty\endcsname{}
%%%%%%%%%%%%%%%%%%%%%%%%%%%%%%%%%%%%%%%%%%%%%%%%%%%%%%%%%%%%%%%%%%%%%%%%%%%%%%%%
%</sty>
%<*cls>
\RequirePackage{ gWmaths, l3keys2e, ifdraft,	}%xparse, expl3
\RequireeTeX\GetFileInfo{\CurrentFile}
%%%%%%%%%%%%%%%%%%%%%%%%%%%%%%%%%%%%%%%%%%%%%%%%%%%%%%%%%%%%%%%%%%%%%%%%%%%%%%%%
\ifx\filename\jobname%\iffalse
%%%%%%%%%%%%%%%%%%%%%%%%%%%%%%%%%%%%%%%%%%%%%%%%%%%%%%%%%%%%%%%%%%%%%%%%%%%%%%%%
%%
%%
%% Copy the following lines to the *top* of your \LaTeX document
%%
%%%%%%%%%%%%%%%%%%%%%%%%%%%%%%%%%%%%%%%%%%%%%%%%%%%%%%%%%%%%%%%%%%%%%%%%%%%%%%%%



\documentclass[english,lecture]{gWmaths}
%!TEX program = pdflatexmk
%!TEX encoding = UTF-8
%!TEX spellcheck = en-US
%!TeX root = COURSE_MAIN.TEX
%!TeX BiB program = biber
%%%%%%%%%%%%%%%%%%%%%%%%%%%%%%%%%%%%%%%%%%%%%%
% Now we're ready to start
%%%%%%%%%%%%%%%%%%%%%%%%%%%%%%%%%%%%%%%%%%%%%%
%%
\date{Fall 2020}  % \LaTeXe uses \today if you don't specify any \date{}
\author{Diana Pell}
\title{TITLE OF DOCUMENT}
%%
\addto\captionsenglish{%
	\renewcommand{\abstractname}{What To Expect}% Heading of Intro section
}
%%


%%%%%%%%%%%%%%%%%%%%%%%%%%%%%%%%%%%%%%%%%%%%%%%%%%%%%%%%%%%%%%%%%%%%%%%%%%%%%%%%
%%
\fi%		Copy the *above* lines to the top of your \LaTeX document
%%
%%
%%%%%%%%%%%%%%%%%%%%%%%%%%%%%%%%%%%%%%%%%%%%%%%%%%%%%%%%%%%%%%%%%%%%%%%%%%%%%%%%
%%%%%%%%%%%%%%%%%%%%%%%%%%%%%%%%%%%%%%%%%%%%%%%%%%%%%%%%%%%%%%%%%%%%%%%%%%%%%%%%
%%
%%
%%%%%%%%%%%%%%%%%%%%%%%%%%%%%%%%%%%%%%%%%%%%%%%%%%%%%%%%%%%%%%%%%%%%%%%%%%%%%%%%
\gW@addGlobalOption{utf8} % All documents are UTF-8 in the 21st century. If not, fix your document.
%%%%%%%%%%%%%%%%%%%%%%%%%%%%%%%%%%%%%%%%%%%%%%%%%%%%%%%%%%%%%%%%%%%%%%%%%%%%%%%%
%% Specify some requirements, but this is mostly redundant...
%\RequireeTeX%necessarily true for all modern \LaTeX
%\RequirePackage{fixltx2e}[2015/01/01]
%\ifpdf{}
%TODO: use \eTeX's \unless form
%\RequirePDF%invalid macro
%\fi
%\RequirePackage%
%	%[2020/05/05] %Request the \LaTeX format in effect as of this date
%	[current]%Use the \LaTeX format as-is.
%	%[latest]%Use the latest \LaTeX format, even if `latexrelease` is newer than the in-box format
%		{latexrelease} % http://mirrors.ctan.org/macros/latex/base/latexrelease.pdf
%%



%%%TODO:
%% http://mirrors.ctan.org/macros/latex/base/slides.pdf
%% http://mirrors.ctan.org/macros/latex209/contrib/eslides/eslides.pdf
%%hypcap
%% http://mirrors.ctan.org/macros/latex/contrib/etoolbox/etoolbox.pdf
%%% \newbool & \newtoggle
%% http://mirrors.ctan.org/macros/latex/contrib/siunitx/siunitx.pdf
%% http://mirrors.ctan.org/macros/latex/contrib/noindentafter/noindentafter.pdf
%% http://ctan.math.washington.edu/tex-archive/support/latexmk/latexmk.pdf
%% http://mirrors.ctan.org/macros/latex/contrib/bookmark/bookmark.pdf
%% https://tex.stackexchange.com/a/31658 use \ifdefined\foo\else\fi
%% https://tex.stackexchange.com/a/472044 \string fuckery


%%%%%%%%%%%%%%%%%%%%%%%%%%%%%%%%%%%%%%%%%%%%%%%%%%%%%%%%%%%%%%%%%%%%%%%%%%%%%%%%
%%%%%%%%%%%%%%%%%%%%%%%%%%%%%%%%%%%%%%%%%%%%%%%%%%%%%%%%%%%%%%%%%%%%%%%%%%%%%%%%
%% Set up our options and defaults
%%%%%%%%%%%%%%%%%%%%%%%%%%%%%%%%%%%%%%%%%%%%%%%%%%%%%%%%%%%%%%%%%%%%%%%%%%%%%%%%

\keys_define:nn { gWmaths }
{
	hyperref	.bool_set:N	= \gW@hyperref	,

	ntheorem	.bool_set:N	= \gW@ntheorem	,

	amsmath	.bool_set:N	= \gW@amsmath	,

	hypdoc	.bool_set:N	= \gW@hypdoc	,

	amsthm	.bool_set:N	= \gW@amsthm	,

	accessibility	.bool_set:N	= \gW@accessibility	,

	axessibility	.bool_set:N	= \gW@axessibility	,

	videolecture	.bool_set:N	= \gW@videolecture	,
	oneinch	.bool_set:N	= \gW@oneinch	,
	
	titlesec	.bool_set:N	= \gW@titlesec	,
	%http://mirrors.ctan.org/macros/latex/contrib/titlesec/titlesec.pdf

	documentClass	.tl_set:N	= \gW@documentClass	,
	documentClass	.default:n	= article	,
	syllabus	.meta:n	= {
		documentClass	= article	,
			}	,
	syllabus	.value_forbidden:n	= true	,
	exam	.meta:n	= {
		documentClass	= exam	,
			}	,
	exam	.value_forbidden:n	= true	,
	lecture	.meta:n	= {
		documentClass	= article
			}	,
	lecture	.value_forbidden:n	= true	,
	course	.meta:n	= {
		documentClass	= article
			}	,
	course	.value_forbidden:n	= true	,
	quiz	.meta:n	= {
		documentClass	= article
			}	,
	quiz	.value_forbidden:n	= true	,
	handout	.meta:n	= {
		documentClass	= article
			}	,
	handout	.value_forbidden:n	= true	,


}

%%%%%%%%%%%%%%%%%%%%%%%%%%%%%%%%%%%%%%%%%%%%%%%%%%%%%%%%%%%%%%%%%%%%%%%%%%%%%%%%
\keys_set:nn { gWmaths }{% Set our default options
	hyperref, 
	ntheorem,
	hypdoc,
	documentClass=article,
	amsmath,
		}
%%%%%%%%%%%%%%%%%%%%%%%%%%%%%%%%%%%%%%%%%%%%%%%%%%%%%%%%%%%%%%%%%%%%%%%%%%%%%%%%
%% Prepare arguments for all the packages
%%%%%%%%%%%%%%%%%%%%%%%%%%%%%%%%%%%%%%%%%%%%%%%%%%%%%%%%%%%%%%%%%%%%%%%%%%%%%%%%
%% Set up modern defaults for encodings:
\PassOptionsToPackage{% `inputenc`
	utf8 % All documents are UTF-8 in the 21st century. If not, fix your document.
		}{inputenc}
\PassOptionsToPackage{% `fontenc`
	T1		% T1 fonts extend the old OT1 format to allow greater than 128 characters
		}{fontenc}
% 
\PassOptionsToPackage{% `mmap`
	noTeX % use unicode codepoints, not TeX names. 
		}{mmap} % Allows copy/pasta from PDF.
% 
\ifdraft{
\PassOptionsToPackage{%
	draft
		}{bookmark}%TODO: replace this with just a damn \DeclareOption
}{}% \ifdraft
% 
\PassOptionsToPackage{%
	atend,%
		}{bookmark}
% 
%% Set up \HyperTeX (hyperref, et al), which is basically the structural core of the document:
\bool_if:nTF \gW@hyperref{
\PassOptionsToPackage{% `hyperref`
	implicit, % Redefine \LaTeX internals, already default
	hypertexnames, % Use guessable names for anchors, already default
	bookmarks, % Write TOC as PDF bookmarks
	driverfallback=hypertex, % if can't autodetect, then just use the most basic

	%hyperfigures, % make figures hyper links
	%hyperfootnotes, % set up hyperlinked footnotes, already default
	%hyperindex, % set up hyperlinked indices, already default

	colorlinks, % set hyperlinks in color
	linkcolor=blue, % default is fucking 'red'

	%hyperindex, % links index entries back to main text, conflicts with hypdoc
	linktoc=all, % link section name *and* page number from toc
	%backref, % link back from bib entries to main text

	%pdfencoding=unicode, % set 'auto' to prefer PDFEncoding over UTF-16; default for \TUTeX
	%pdflang=en-us, % `hyperxmp` automatically sets this from the language of `babel`
	pdfusetitle, % use \title{} for PDF metadata
	pdfdisplaydoctitle, % set PDF display \title{} in UI title bar
	pdfpagelabels, % tag pages with page numbers from \LaTeX ("ii" or whatever) 
	keeppdfinfo % `hyperxmp` strips Author and Keywords by default; don't do that.
		}{hyperref}
%
%% Inform *all* packages that `hyperref` is in use:
\gW@addGlobalOption{hyperref}
%%
%\PassOptionsToPackage{% `xcolor`
%	hyperref % Support the `hyperref` package in terms of color expressions by defining additional keys
%		}{xcolor}
%\PassOptionsToPackage{% `ntheorem`
%	hyperref, % Support the `hyperref` package
%		}{ntheorem}
}{}% \@gWhyperref

%
%% Set up `scrbase` not to argue with iftex.
\PassOptionsToPackage{% `scrbase`
	internalonly % don't define macros to conflict with `iftex`
		}{scrbase} % KOMA-Script base package


%
%% Set up documentation utilities from `hyperref` documentation package.
\bool_if:nTF \gW@hypdoc{
\PassOptionsToPackage{% `hyperref`
	hyperindex=false, % do *not* link page numbers from index...conflict w/ hypdoc...
		}{hyperref}
}{}%\gW@hypdoc

%
%% Set up colors now that we have color printers and video screens...
\PassOptionsToPackage{% `xcolor`
	xcdraw, % use PS/PDF commands to draw frames and boxes, in dvips, pdftex, dvipdfm
	fixinclude, % prevents dvips from explicitly resetting current color to black before actually inserting an .eps file
	%fixpdftex % Load the `pdfcolmk` package; useless stub
	dvipsnames, % Load a set of predefined colors.
	svgnames, % Load a set of predefined colors according to SVG 1.1.
	x11names*, % Load a set of predefined colors according to Unix/X11, delayed.
	table, % Load the `colortbl` package
		}{xcolor}

%
%% Set up a much nicer set of theorem environments:
\bool_if:nTF \gW@ntheorem{
\PassOptionsToPackage{% `ntheorem`
	thmmarks, % enables the automatical placement of endmarks
	thref, % enables the extended reference features
	standard % load predefined environments
		}{ntheorem}
%
%% Inform *all* packages that `ntheorem` is in use:
\gW@addGlobalOption{ntheorem}
%%
%\PassOptionsToPackage{% `empheq`
%	ntheorem % cope with `ntheorem`
%		}{empheq}
\PassOptionsToClass{% `beamer`
	noamsthm % prevent from loading `amsthm`, which conflicts with `ntheorem`
		}{beamer}
}{}%\gW@ntheorem

%
%% Set up the most common mathematics package:
\bool_if:nTF \gW@amsmath{
\PassOptionsToPackage{% `amsmath`
		}{amsmath}

%
%% Inform *all* packages that `amsmath` is in use:
\gW@addGlobalOption{amsmath}
%%
%\PassOptionsToPackage{% `ntheorem`
%	amsmath, % cope with `amsmath`
%		}{ntheorem}
}{}

%
%% Set up formatting for telephone numbers:
\PassOptionsToPackage{% `phonenumbers`
	country=US, % switch defaults from DE to US.
	foreign=international, % Use sensible defaults for full ITU numbers.
	home-country=none, % Do not strip the country code (default).
	home-area-code=none, % Do *not* strip area code, ever (default).
	link=on % clickable tel: link (default).
		}{phonenumbers}

%
%% Set up a nicer quotation package:
\PassOptionsToPackage{% `csquotes`
	autostyle=try, % use `babel` if available
	%autopunct % look ahead for trailing punctuation to move inside marks
		}{csquotes}



\PassOptionsToPackage{% `currfile`
	abspath, %-recorder (default in latexmk)
	realmainfile,
	parent,
	%parents,
		}{currfile}



\PassOptionsToClass{% `beamer`
	t % ?
		}{beamer}




%\PassOptionsToPackage{% `babel`
	%english % The language can be tagged on the \documentclass declaration!
		%}{babel}

\PassOptionsToPackage{%
	%heightrounded % round off to avoid overfull alrgn;earujs
		}{geometry} 


%\PassOptionsToPackage{%
%\PassOptionsToPackage{%




%%%%%%%%%%%%%%%%%%%%%%%%%%%%%%%%%%%%%%%%%%%%%%%%%%%%%%%%%%%%%%%%%%%%%%%%%%%%%%%%
%% Input encoding, font loading, Unicode
%% ref: https://tex.stackexchange.com/a/64457
%% ref: https://tex.stackexchange.com/a/677
%% ref: https://tex.stackexchange.com/a/147206
%% ref: https://tex.stackexchange.com/a/2869
%% ref: https://tex.stackexchange.com/a/44701 
%% ref: https://tex.stackexchange.com/a/413046
%% ref: https://tex.stackexchange.com/a/54077
%% ref: http://www.texfaq.org/FAQ-why-inp-font
%% ref: https://tex.stackexchange.com/a/78103
%% ref: https://robjhyndman.com/hyndsight/xelatex/
%% ref: http://ipagwww.med.yale.edu/latex/hyperref.pdf % page size
%% ref: https://tex.stackexchange.com/q/1863
%% ref: https://tex.stackexchange.com/questions/331597/trouble-transitioning-from-paralist-to-enumitem
%% https://tex.stackexchange.com/a/202301


%todo etoolbox http://mirrors.ctan.org/macros/latex/contrib/etoolbox/etoolbox.pdf

%todo ifthen http://mirrors.ctan.org/macros/latex/base/ifthen.pdf

\let\@classoptionslist\gW@classoptionslist
% ref: https://www.overleaf.com/learn/latex/Writing_your_own_class
% ref: https://www.latex-project.org/help/documentation/clsguide.pdf

\bool_if:nTF \gW@accessibility{
	\requirePDFTeX
}{}
%\bool_if:nTF \gW@axessibility{}{}
%\bool_if:nTF \gW@videolecture{}{}

%declare option for setting margin to 1"
%\bool_if:nTF \gW@oneinch{}{}


% The \LaTeX2e way:
%\newif\ifgW@myoption% create a flag for this option
%\DeclareOption{someoption}{\gW@myoptiontrue}% if the option is set, then raise the flag


\ProcessKeysOptions{gWmaths}

\DeclareOption*{\PassOptionsToClass{\CurrentOption}{\gW@documentClass}}
\ProcessOptions\relax

\ifpdf
	\pdfminorversion=6 %fucking hack for `accessibility`
%	\pdf_version_gset:n {6}
%	\pdf_version_min_gset:n {6}
\fi


%\RequirePackage% Automatically configure package load order.
%	[2018/04/29]
%		{pkgloader} % http://mirrors.ctan.org/macros/latex/contrib/pkgloader/pkgloader.pdf
%
\providecommand{\LoadPackagesNow}{}% reduce code churn by providing a stub for running without `pkgloader`


\ifTUTeX\else% engine is not unicode-native
	\RequirePackage[utf8]{inputenc}[2015/01/01]
	% When using older engines, this tells the engine to read all files as UTF-8.
	% XeTeX and LuaTeX always read UTF-8 and ignore this package. 
	% LaTeX circa 2018 and newer default to UTF-8 as well. 
\fi


%%%%%%%%%%%%%%%%%%%%%%%%%%%%%%%%%%%%%%%%%%%%%%%%%%%%%%%%%%%%%%%%%%%%%%%%%%%%%%%%
%% Load the base document class
\LoadClass{\gW@documentClass}
%%

%%%%%%%%%%%%%%%%%%%%%%%%%%%%%%%%%%%%%%%%%%%%%%%%%%%%%%%%%%%%%%%%%%%%%%%%%%%%%%%%
%% Begin actually loading packages
%%%%%%%%%%%%%%%%%%%%%%%%%%%%%%%%%%%%%%%%%%%%%%%%%%%%%%%%%%%%%%%%%%%%%%%%%%%%%%%%
% See https://github.com/mhelvens/latex-pkgloader/blob/master/pkgloader-recommended.sty
%%
\RequirePackage{% `hyperxmp` required *before* `hyperref`
	babel,		% http://mirrors.ctan.org/macros/latex/required/babel/base/babel.pdf
	hyperxmp		% http://mirrors.ctan.org/macros/latex/contrib/hyperxmp/hyperxmp.pdf
		}%	% babel cannot set the pdf-lang option for hyperref, but it can through hyperxmp!


\RequirePackage{%
	amsfonts,		% http://mirrors.ctan.org/fonts/amsfonts/doc/amsfndoc.pdf
	amssymb,		% http://mirrors.ctan.org/fonts/amsfonts/doc/amssymb.pdf
	amsmath,		% http://mirrors.ctan.org/macros/latex/required/amsmath/amsldoc.pdf
	%mathtools,		%http://mirrors.ctan.org/macros/latex/contrib/mathtools/mathtools.pdf
%%don't use: http://mirrors.ctan.org/macros/latex/required/amscls/doc/amsthdoc.pdf
		}%

\RequirePackage{%
	venndiagram
		}%https://tex.stackexchange.com/a/381249

\bool_if:nTF \gW@ntheorem{
\RequirePackage{% improves theorem-like environments, conflicts with `amsthm`, must load *after* `babel` and *after* amsmath and *after* mathtools/empheq
	ntheorem		% http://mirrors.ctan.org/macros/latex/contrib/ntheorem/ntheorem.pdf 
		}	%https://tex.stackexchange.com/a/5633
}{}%\gW@ntheorem



\bool_if:nTF \gW@oneinch{ % if the flag is set, then do some thing
	\AtEndOfClass{
		\geometry{
			%letterpaper, % screen % try to get this to auto-detect or something so \documentclass[a4paper,landscape]{gWmaths} works
			margin=1in
				}
			}
}{}%\gW@oneinch

%\ifgW@myoption % if the flag is set, then do some thing
%   code active only for this option
%\fi

\ifXeTeX
	%% docs: http://mirrors.ctan.org/info/xetexref/xetex-reference.pdf
\fi


\RequirePackage% load *before* `fontspec`
	{graphicx} % http://mirrors.ctan.org/macros/latex/required/graphics/grfguide.pdf
		% docs say to only load one, and package/classes should  `s` not `x` as the user may s->x but not back
		% literally the only difference is the `keyval` package, therefore just use `x`.

\ifTUTeX % "Unicode TeX" matches LuaTeX and/or XeTeX
	\RequirePackage% must load *after* maths fonts (specifically euler)
		{fontspec} % http://mirrors.ctan.org/macros/unicodetex/latex/fontspec/fontspec.pdf
	\defaultfontfeatures{Ligatures=TeX} % Make ASCII nicer
	
	%% Post-2017, don't use EU1/EU2 font encoding (nor T1, nor TS1, nor OT1) 
	%% Post-2017, don't use xunicode package
	%% Post-2017, default `tuenc` package uses TU font encoding, full unicode.
\else % pdfTeX or older
	\RequirePackage{%
		mmap,	% `mmap` must load *after* `hyperref` % https://ctan.org/pkg/mmap
		fontenc,	% `fontenc` must load *after* `mmap`/`cmap`	% `fontenc` specifies which font format to import.
			}%
	% T1 fonts extend the old OT1 format to allow greater than 128 characters
	% `mmap` is `cmap` plus some math-specifics. 
	% `cmap` extends default character mapping to include unicode codepoints.
	% something about ASCII or Unicode fonts: ...Package[noTeX]{mmap}
\fi



%\RequirePackage[%
	%% docs: http://mirrors.ctan.org/macros/latex/contrib/standalone/standalone.pdf
%	subpreambles=false % do not automatically collect preambles
%		]{standalone}
%\documentclass[class=gWmaths,crop=false]{standalone}
%\standaloneconfig{}

%\RequirePackage[%
	%activeospeccharacters, % don't fuck with < and > 
	%notheorems?
		%]{beamerarticle}
%\documentclass[ignorenonframetext]{beamer}
%\documentclass[trans]{beamer}
%Note: When using \include or \input commands, conversions of modes must be controlled manually. See Section 21.3 for details
%\mode<article>{\usepackage{fullpage}}
%\mode<presentation>{\usetheme{Berlin}}
%\subsection<article>{Article-Only Section}
%\only<article>{\item This particular item is only part of the article version.}
%\item<presentation:only@0> This text is also only part of the article.
%The command \setjobnamebeamerversion{main.beamer} tells the article version where to find the presentation version. This is necessary if you wish to include slides from the presentation version in an article as figures.
%By adjusting the frame template, you can “mimic” the appearance of frames typeset by beamer in your articles.


%https://tex.stackexchange.com/a/5231

%pdflatex --jobname=Cookies %\jobname


% These packages are either required for some of our features, or are just really nice to have set up:
\RequirePackage% `fancyhdr` may have issues with `accessibility`, must load *before* `hyperref`
	{fancyhdr} % http://mirrors.ctan.org/macros/latex/contrib/fancyhdr/fancyhdr.pdf

\bool_if:nTF \gW@videolecture{}{
	\RequirePackage% `paralist` redefines {itemize}, {enumerate} environments
		{paralist} % http://mirrors.ctan.org/macros/latex/contrib/paralist/paralist.pdf
%\RequirePackage[ % `enumitem` redefines {itemize}, {enumerate} environments
	% `enumitem` is substantially more complex than `paralist`
	%% docs: http://mirrors.ctan.org/macros/latex/contrib/enumitem/enumitem.pdf
	%shortlabels, % compatible with `enumerate`
	%unboxed % avoid problems nesting environments
		%]{enumitem}
}%\gW@videolecture



\ifTUTeX % Unicode TeX
	\RequirePackage % Use unicode symbols
		{unicode-math} % http://mirrors.ctan.org/macros/unicodetex/latex/unicode-math/unicode-math.pdf
\fi

%\RequirePackage% `pdfpages` improves inserting PDF clippings over graphicx
%	{pdfpages} % http://mirrors.ctan.org/macros/latex/contrib/pdfpages/pdfpages.pdf

\RequirePackage{%
	xcolor,	% http://mirrors.ctan.org/macros/latex/contrib/xcolor/xcolor.pdf
	phonenumbers,	% http://mirrors.ctan.org/macros/latex/contrib/phonenumbers/doc/phonenumbers-en.pdf
	csquotes,	% http://mirrors.ctan.org/macros/latex/contrib/csquotes/csquotes.pdf
	geometry,	% http://mirrors.ctan.org/macros/latex/contrib/geometry/geometry.pdf
	filemod,	% http://mirrors.ctan.org/macros/latex/contrib/filemod/filemod.pdf
		}

\bool_if:nTF \gW@videolecture{}{
	\RequirePackage% un-dumb \maketitle
		{titling} % http://mirrors.ctan.org/macros/latex/contrib/titling/titling.pdf
}%\gW@videolecture

%define new \listofXs
% docs: http://mirrors.ctan.org/macros/latex/contrib/tocloft/tocloft.pdf

\bool_if:nTF \gW@accessibility{ %\ifPDFTeX % not XeTeX nor LuaTeX
	% This package is the magic that associates, structures, and tags the TeX sources to show up in the PDF:
	\requirePDFTEX % `accessibility` is badly hacked together and quite old...
	\RequirePackage[tagged]{accessibility}
}{}%\gW@accessibility

\ifXeTeX\else% seems like XeTeX is actually kind of a hack and maybe not much future...
	\RequirePackage[
		%% docs: 
			]{tagpdf} % official work-in-progress package for future integrated tagging
	\RequirePackage[%
		%% docs: http://mirrors.ctan.org/macros/latex/contrib/axessibility/axessibility.pdf
			]{axessibility}
\fi

%http://mirrors.ctan.org/macros/latex/contrib/titlesec/titlesec.pdf % too complex

\RequirePackage{% `hyperref` claims to be about hypertext, but it's actually a significant structural support!
	%% NOTE: https://tex.stackexchange.com/q/1863
	hyperref,		% http://mirrors.ctan.org/macros/latex/contrib/hyperref/doc/manual.pdf
	bookmark,		% http://mirrors.ctan.org/macros/latex/contrib/bookmark/bookmark.pdf
	%pdfcomment,	% http://mirrors.ctan.org/macros/latex/contrib/pdfcomment/doc/pdfcomment.pdf
		}% %also: http://mirrors.ctan.org/macros/latex/contrib/hyperref/doc/manual.html#x1-520009

\bool_if:nTF \gW@videolecture{}{
	\RequirePackage%
		{hypdoc} % http://mirrors.ctan.org/macros/latex/contrib/oberdiek/hypdoc.pdf
}%\gW@videolecture

%\RequirePackage[ % `pdfcomment` requires `hyperref` to be loaded first
	%% docs: http://mirrors.ctan.org/macros/latex/contrib/pdfcomment/doc/pdfcomment.pdf
	%linewidth = 1 % TODO: what is `linewidth` for?
		%]{pdfcomment} % http://mirrors.ctan.org/macros/latex/contrib/pdfcomment/doc/pdfcomment.pdf

%hack for some releases of TeX Live which disable auto-detection
\ifXeTeX\geometry{xetex}\fi% 
\ifLuaTeX\geometry{luatex}\fi% 

%\RequirePackage% \currfilename
%	{currfile} % http://mirrors.ctan.org/macros/latex/contrib/currfile/currfile.pdf

\ifLuaTeX
%	\RequirePackage{pdftexcmds}
\fi



\RequirePackage%
	{filehook,currfile}%
		[2020/09/29]%



%\RequirePackage%
%	{ifthen} % http://mirrors.ctan.org/macros/latex/base/ifthen.pdf

%%%%%%%%%%%%%%%%%%%%%%%%%%%%%%%%%%%%%%%%%%%%%%%%%%%%%%%%%%%%%%%%%%%%%%%%%%%%%%%%
% Ugly hack, because of the unfortunate deprecation of 'n' to 'c' conversion,
% plus the fact that kernel errors, even non-fatal ones, cannot be redirected.
% Will think of a better solution at some point, but not now.
%\let\__withargs_docs_old_kernel_msg_error:nnnnnn\__kernel_msg_error:nnnnnn
%\def\__kernel_msg_error:nnnnnn#1#2{
%  \str_if_eq:nnTF { #1/#2 } { kernel/deprecated-variant }{
%    \__kernel_msg_warning:nnnnnn{#1}{#2}
%  }{
%    \__withargs_docs_old_kernel_msg_error:nnnnnn{#1}{#2}
%  }
%}
%\fi
%%%%%%%%%%%%%%%%%%%%%%%%%%%%%%%%%%%%%%%%%%%%%%%%%%%%%%%%%%%%%%%%%%%%%%%%%%%%%%%%

\LoadPackagesNow% `pkgloader`


\@ifundefined{GetFileInfo}{}{% https://tex.stackexchange.com/a/30486
	%http://mirrors.ctan.org/macros/latex/base/ltfilehook-doc.pdf
	\GetFileInfo{\CurrentFile}
	\hypersetup{
		pdfsubject={\fileinfo}, % this takes the name/description of the *class*...
			}

	\AtEndOfClass{%
		%\GetFileInfo{\jobname.tex} % read \ProvidesFile{}[] from the ultimate TeX
		%\listfiles % track loaded files in this job
		%\@getclass
		\date{\filemodprintdate{\filename}}
			}
		}% \GetFileInfo


\hypersetup{%
	%pdflang={en-US}, % `hyperxmp` will ask `babel` for the language code!
	%hyperindex % turn this shit back on, JFC.
		}

\AtBeginDocument{%
	\listfiles % track loaded files in this job
	%\date{omg}
	\hypersetup{%
		%pdfdate={\pdffilemoddate{\jobname.tex}}, % 2020 or 2020-01 or 2020-01-04
		pdfcreationdate={\pdffilemoddate{\jobname.tex}},
		pdfrendition=default, %screen,
		%pdfsubject={},
		%pdfsubtitle={},
		%pdfurl={}, % full path to *this* PDF
		pdfinfo={%
			%CreationDate={\pdffilemoddate{\jobname.tex}}
				},
		%pdfsource={\jobname.tex} % default
		pdfuapart=1 % that's the whole point, but we're lacking...
			}
		}


%%%%%%%%%%%%%%%%%%%%%%%%%%%%%%%%%%%%%%%%%%%%%%%%%%%%%%%%%%%%%%%%%%%%%%%%%%%%%%%%
%% preserve document metadata...
\AtEndPreamble
{
	%\let\doctitle\@title
	%\let\docauthor\@author
	%\let\docthanks\@thanks % this is bjorked due to missing macros
	%\let\docdate\@date
}




\providecommand*{\q}[2][]{\blockquote[#1][]{#2}}
	% Short-hand for quotes via `csquotes`
	% TODO: \blockcquote
	% \providencommand can't more than one optional argument...
	%% ref: https://tex.stackexchange.com/a/29975
	%% ref: https://www.overleaf.com/learn/latex/Commands
	%% ref: https://tex.stackexchange.com/questions/321435/newcommand-and-renewcommand-difficulty-in-class-file
	%% ref: https://stackoverflow.com/questions/1812214/latex-optional-arguments
	%% ref: https://tex.stackexchange.com/a/1057
	

\RequirePackage[%
	%% docs: http://mirrors.ctan.org/macros/latex/required/tools/xspace.pdf
		]{xspace}
\AtBeginDocument{%
	\hologoSetup{%
		break=false
			}
		}


\RequirePackage[%
	%% docs: http://mirrors.ctan.org/macros/latex/contrib/layouts/layman.pdf
		]{layouts}
\providecommand{\drawlayouts}	{%
	\drawdimensionstrue
	\printinunitsof{in}
	\pagediagram   % draws diagram with all layout vernacular identified (except bottom margin)
	\pagevalues
	\currentpage
		}


%%%%%%%%%%%%%%%%%%%%%%%%%%%%%%%%%%%%%%%%%%%%%%%%%%%%%%%%%%%%%%%%%%%%%%%%%%%%%%%%
% Parts, Chapters, Sections, Subsections, Subsubsections, Paragraphs, Subparagraphs
%%
% https://www.overleaf.com/learn/latex/sections_and_chapters
% https://en.wikibooks.org/wiki/LaTeX/Document_Structure#Sectioning_commands
%%

% Set our header depth to maximum 3:
\setcounter{secnumdepth}{3} % '3' is default, and numbers to subsubsections not paragraphs
\setcounter{tocdepth}{2} % number to subsection, not subsubsection
%%%%%%%%%%%%%%%%%%%%%%%%%%%%%%%%%%%%%%%%%%%%%%%%%%%%%%%%%%%%%%%%%%%%%%%%%%%%%%%%



\bool_if:nTF \gW@videolecture{
	\providecommand*{\theorembreak}{\usebeamertemplate{theorem end}\framebreak\usebeamertemplate{theorem begin}}
}{}%\gW@videolecture

%%%%%%%%%%%%%%%%%%%%%%%%%%%%%%%%%%%%%%%%%%%%%%%%%%%%%%%%%%%%%%%%%%%%%%%%%%%%%%%%


\let\IfPackageLoaded\@ifpackageloaded
%\let\IfPackageLoaded\ltx@ifpackageloaded
%https://tex.stackexchange.com/a/484092
%http://mirrors.ctan.org/macros/generic/ltxcmds/ltxcmds.pdf


\NewDocumentCommand \gWmathssetup { m } {
	\keys_set:nn { gWmaths } { #1 }
}


\ExplSyntaxOff


%%%%%%%%%%%%%%%%%%%%%%%%%%%%%%%%%%%%%%%%%%%%%%%%%%%%%%%%%%%%%%%%%%%%%%%%%%%%%%%%
%  Begin user defined commands

\providecommand{\map}[1]{\xrightarrow{#1}}

\providecommand{\bc}{\mathbb C}
\providecommand{\bF}{\mathbb F}
\providecommand{\bH}{\mathbb H}
\providecommand{\bn}{\mathbb N}
\providecommand{\bz}{\mathbb Z}
\providecommand{\bp}{\mathbb{P}}
\providecommand{\bq}{\mathbb Q}
\providecommand{\br}{\mathbb R}


\providecommand{\zbar}{\overline{\mathbb{Z}}}
\providecommand{\qbar}{\overline{\mathbb{Q}}}

\providecommand{\la}{\langle}
\providecommand{\ra}{\rangle}
\providecommand{\lra}{\longrightarrow}
\providecommand{\hra}{\hookrightarrow}
\providecommand{\bs}{\backslash}

\providecommand{\al}{\alpha}
\providecommand{\be}{\beta}

\DeclareMathOperator{\Aut}{Aut}
\DeclareMathOperator{\Aff}{Aff}
\DeclareMathOperator{\End}{End}
\DeclareMathOperator{\Hom}{Hom}
\DeclareMathOperator{\im}{im}

%\renewcommand{\labelenumi}{(\alphaph{enumi})}

\providecommand{\blankpage}	{%
      \clearpage%
      \thispagestyle{empty}%
      \addtocounter{page}{-1}%
      \null%
      \clearpage}

%  End user defined commands
%%%%%%%%%%%%%%%%%%%%%%%%%%%%%%%%%%%%%%%%%%%%%%%%%%%%%%%%%%%%%%%%%%%%%%%%%%%%%%%%



%%%%%%%%%%%%%%%%%%%%%%%%%%%%%%%%%%%%%%%%%%%%%%%%%%%%%%%%%%%%%%%%%%%%%%%%%%%%%%%%
% These establish different environments for stating Theorems, Lemmas, Remarks, etc.

\newtheorem{Pf}{Proof}

\newtheorem{Thm}{Theorem}
\newtheorem{Prop}[Thm]{Proposition}
\newtheorem{Lem}[Thm]{Lemma}
\newtheorem{Cor}[Thm]{Corollary}

\theoremstyle{definition}
\newtheorem{Def}[Thm]{Definition}

\theoremstyle{remark}
\newtheorem{Rem}[Thm]{Remark}
\newtheorem{Ex}[Thm]{Example}

\theoremstyle{definition}
\newtheorem{Exercise}{Exercise}

\newenvironment{Solution}{\noindent{\it Solution.}}


% End environments 
%%%%%%%%%%%%%%%%%%%%%%%%%%%%%%%%%%%%%%%%%%%%%%%%%%%%%%%%%%%%%%%%%%%%%%%%%%%%%%%%%









%\begin{filecontents*}{latexmkrc}
%\end{filecontents}



%%%%%%%%%%%%%%%%%%%%%%%%%%%%%%%%%%%%%%%%%%%%%%%%%%%%%%%%%%%%%%%%%%%%%%%%%%%%%%%%
\ifx\filename\jobname%%%%%%%%%%%%%%%%%%%%%%%%%%%%%%%%%%%%
%%%%%%%%%%%%%%%%%%%%%%%%%%%%%%%%%%%%%%%%%%%%%%%%%%%%%%%%%%%%%%%%%%%%%%%%%%%%%%%%
\tagpdfsetup{}
\hypersetup{
	colorlinks=true,
	linkcolor=blue,
	filecolor=magenta,      
	urlcolor=cyan,
		}%\hypersetup
\urlstyle{
	%same, % Use the 'same' style as surrounding text, default is monospace font
		}%\urlstyle
\bookmarksetup{}
\geometry{}
\graphicspath{}

\GetFileInfo{\CurrentFile}
\title{\fileinfo}
\date{\filedate}



\begin{document}


\end{document}

%%%%%%%%%%%%%%%%%%%%%%%%%%%%%%%%%%%%%%%%%%%%%%%%%%%%%%%%%%%%%%%%%%%%%%%%%%%%%%%%
\fi%%%%%%%%%%%%%%%%%%%%%%%%%%%%%%%%%%%%%%%%%%%%%
%%%%%%%%%%%%%%%%%%%%%%%%%%%%%%%%%%%%%%%%%%%%%%%%%%%%%%%%%%%%%%%%%%%%%%%%%%%%%%%%
